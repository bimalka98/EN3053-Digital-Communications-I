%  -----------------------------------------------------------------------------
%  Author         : Bimalka Piyaruwan Thalagala
%  GitHub         : https://github.com/bimalka98
%  -----------------------------------------------------------------------------

\documentclass[a4paper,11pt]{article}%,twocolumn
\input{settings/packages}
\input{settings/page}
\input{settings/macros}
\usepackage[ framed, numbered]{matlab-prettifier}%framed,%
\usepackage{listings}
\usepackage{pythonhighlight}
\usepackage{pdfpages}

\begin{document}

\begin{titlepage}
\center % Center everything on the page

%-------------------------------------------------------------------------------------
%	HEADING SECTIONS
%------------------------------------------------------------------------------------
\textbf{\large Department of Electronic and Telecommunication Engineering}\\[0.5cm]
\textbf{\Large University of Moratuwa, Sri Lanka}\\[1cm]
\textbf{\large EN3053 - Digital Communications - I}\\[2cm]
\includegraphics[width=0.3\textwidth]{figures/uomlogo}\\[2cm]

	
%-------------------------------------------------------------------------------------
%	TITLE SECTION
%------------------------------------------------------------------------------------
\textbf{\Huge Lab Assignment}\\[0.5cm]
\textbf{\Large Eye diagrams and Equalization }\\


%----------------------------------------------------------------------------------------
%	MEMBERS SECTION
%----------------------------------------------------------------------------------------
\vfill % Fill the rest of the page with whitespace

\textbf{\large Submitted by}\\[0.5cm]
\begin{minipage}{0.2\textwidth}
	\begin{flushleft}
		{\large Thalagala B.P.}\\[4mm]
		{\large Samarasinghe P.}\\[4mm]

		
		
	\end{flushleft}
\end{minipage}
\hspace{5mm}
\begin{minipage}{0.2\textwidth}
	\begin{flushright}
		{\large 180631J }\\[4mm]
		{\large 180554B }\\[4mm]

	\end{flushright}
\end{minipage}\\[1.5cm]

%----------------------------------------------------------------------------------------
%	DATE SECTION
%----------------------------------------------------------------------------------------

\textbf{\large Submitted on}\\[0.5cm]
\textbf{\Large \today} % Date, change the \today to a set date if you want to be precise

%----------------------------------------------------------------------------------------



\end{titlepage}

\tableofcontents
\listoffigures
\listoftables
\vfill


\begin{center}
	\textit{\textbf{Note:}}
\textit{{\tt MATLAB R2018a} of the MathWorks Inc. is used for the implementation.}
\end{center}
\pagebreak

%%-----------------------------------------------------------------------
\section{Task 1}

Please note that for the MATLAB implementation bit rate(bits/second) of the generator was assumed to be 10. As we consider BPSK for the Task 1 and 2, the symbol rate(symbols/second) is remain the same as the bit rate of the generator.

\subsection{Generation of an Impulse Train Representing BPSK Symbols}
Binary data of the generator $D \in \{0, 1\}$ is mapped in to an impulse train according to the following function where $A(Amplitude)$  of the impulse was taken as 1 in the MATLAB implementation.
\[
amplitude ~of~ the~ k^{th}~ impulse = \begin{cases}
	+A & if~ D = 1\\
	-A & if~ D = 0
\end{cases}
\]

\subsection{Transmit Signal}
\subsection{Sinc function as the Impulse response}

Function {\tt sinc(t)} in MATLAB is defined as follows. 
\[
{\tt sinc(t)} = \begin{cases}
	\frac{sin(\pi.t)}{\pi.t} & if~t \neq 0\\
	1 & if~ t = 1
\end{cases}
\]

In order to generate  a sinc pulse that aligns with our time scale the function argument should be given as mentioned below. Where $T_b$ is the separation between
successive transmitted pulses.

\[
sinc~pulse = {\tt sinc(\frac{t}{T_b})}
\]


\subsection{Eye Diagram}

\section{Task 2}
Task 2 is a repetition of the Task 1, but in the presence of additive white Gaussian noise (AWGN). Variance of the noise $N_0/2$ was set such that the Power efficiency $\gamma = E_b/N_0 = 10 ~dB$. Where $E_b$ is the average bit energy and $N_0$ is the noise power spectral density.

\[
\begin{split}
	\gamma~in~dB &= 10.\log_{10}(E_b/N_0)\\
	\gamma/10 &= \log_{10}(E_b/N_0)\\
	10^{\gamma/10} &= E_b/N_0\\
	N_0 &= \frac{E_b}{10^{\gamma/10}}
\end{split}
\]

\[\therefore ~\sigma_{noise}^2 = N_0/2 = \frac{E_b}{2 \times 10^{\gamma/10}}  \]
Assuming the $P(D = 0) = P(D = 1) = 1/2$ as we consider sufficient large amount of bits in the initial bit stream,
\[
E_b = \frac{1}{2} \times \left[ (+A)^2 + (-A)^2 \right]
 = \frac{1}{2} \times \left[ (+1)^2 + (-1)^2 \right]
 = 1
\]
%---------------------------------------------------------------------------
\end{document}
